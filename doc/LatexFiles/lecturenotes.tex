
% LaTeX Beamer file automatically generated from DocOnce
% https://github.com/doconce/doconce

%-------------------- begin beamer-specific preamble ----------------------

\documentclass{beamer}

\usetheme{red_plain}
\usecolortheme{default}

% turn off the almost invisible, yet disturbing, navigation symbols:
\setbeamertemplate{navigation symbols}{}

% Examples on customization:
%\usecolortheme[named=RawSienna]{structure}
%\usetheme[height=7mm]{Rochester}
%\setbeamerfont{frametitle}{family=\rmfamily,shape=\itshape}
%\setbeamertemplate{items}[ball]
%\setbeamertemplate{blocks}[rounded][shadow=true]
%\useoutertheme{infolines}
%
%\usefonttheme{}
%\useinntertheme{}
%
%\setbeameroption{show notes}
%\setbeameroption{show notes on second screen=right}

% fine for B/W printing:
%\usecolortheme{seahorse}

\usepackage{pgf}
\usepackage{graphicx}
\usepackage{epsfig}
\usepackage{relsize}

\usepackage{fancybox}  % make sure fancybox is loaded before fancyvrb

\usepackage{fancyvrb}
\usepackage{minted} % requires pygments and latex -shell-escape filename
%\usepackage{anslistings}
%\usepackage{listingsutf8}

\usepackage{amsmath,amssymb,bm}
%\usepackage[latin1]{inputenc}
\usepackage[T1]{fontenc}
\usepackage[utf8]{inputenc}
\usepackage{colortbl}
\usepackage[english]{babel}
\usepackage{tikz}
\usepackage{framed}
% Use some nice templates
\beamertemplatetransparentcovereddynamic

% --- begin table of contents based on sections ---
% Delete this, if you do not want the table of contents to pop up at
% the beginning of each section:
% (Only section headings can enter the table of contents in Beamer
% slides generated from DocOnce source, while subsections are used
% for the title in ordinary slides.)
\AtBeginSection[]
{
  \begin{frame}<beamer>[plain]
  \frametitle{}
  %\frametitle{Outline}
  \tableofcontents[currentsection]
  \end{frame}
}
% --- end table of contents based on sections ---

% If you wish to uncover everything in a step-wise fashion, uncomment
% the following command:

%\beamerdefaultoverlayspecification{<+->}

\newcommand{\shortinlinecomment}[3]{\note{\textbf{#1}: #2}}
\newcommand{\longinlinecomment}[3]{\shortinlinecomment{#1}{#2}{#3}}

\definecolor{linkcolor}{rgb}{0,0,0.4}
\hypersetup{
    colorlinks=true,
    linkcolor=linkcolor,
    urlcolor=linkcolor,
    pdfmenubar=true,
    pdftoolbar=true,
    bookmarksdepth=3
    }
\setlength{\parskip}{0pt}  % {1em}

\newenvironment{doconceexercise}{}{}
\newcounter{doconceexercisecounter}
\newenvironment{doconce:movie}{}{}
\newcounter{doconce:movie:counter}

\newcommand{\subex}[1]{\noindent\textbf{#1}}  % for subexercises: a), b), etc

%-------------------- end beamer-specific preamble ----------------------

% Add user's preamble




% insert custom LaTeX commands...

\raggedbottom
\makeindex

%-------------------- end preamble ----------------------

\begin{document}

% matching end for #ifdef PREAMBLE

\newcommand{\exercisesection}[1]{\subsection*{#1}}



% ------------------- main content ----------------------



% ----------------- title -------------------------

\title{Artificial intelligence and machine learning in physics, introduction }

% ----------------- author(s) -------------------------

\author{Morten Hjorth-Jensen\inst{1}}
\institute{Department of Physics and Center for Computing in Science Education, University of Oslo, Norway\inst{1}}
% ----------------- end author(s) -------------------------

\date{Geilo Winter school, March 10-20, 2025
% <optional titlepage figure>
% <optional copyright>
}

\begin{frame}[plain,fragile]
\titlepage
\end{frame}

\begin{frame}[plain,fragile]
\frametitle{What is this talk about?}

\begin{block}{}
The main emphasis is to give you a short and pedestrian introduction to the whys and hows we can use (with several examples) machine learning methods
in physics. And why this could (or should) be of interest. 
\end{block}

\begin{block}{}
These slides and more at \href{{http://mhjensenseminars.github.io/MachineLearningTalk/doc/pub/Geiloschool/}}{\nolinkurl{http://mhjensenseminars.github.io/MachineLearningTalk/doc/pub/Geiloschool/}}
\end{block}
\end{frame}

\begin{frame}[plain,fragile]
\frametitle{Thanks to many}

Jane Kim (MSU), Julie Butler (MSU), Patrick Cook (MSU), Danny Jammooa (MSU), Daniel Bazin (MSU), Dean Lee (MSU), Witek Nazarewicz (MSU), Michelle Kuchera (Davidson College), Even Nordhagen (UiO), Robert Solli (UiO, Expert Analytics), Bryce Fore (ANL), Alessandro Lovato (ANL), Stefano Gandolfi (LANL), Francesco Pederiva (UniTN), and Giuseppe Carleo (EPFL). 
Niyaz Beysengulov and Johannes Pollanen (experiment, MSU); Zachary Stewart, Jared Weidman, and Angela Wilson (quantum chemistry, MSU)
Jonas Flaten, Oskar, Leinonen, Øyvind Sigmundson Schøyen, Stian Dysthe Bilek, and Håkon Emil Kristiansen (UiO). Marianne Bathen and Lasse Vines (experiments (UiO). Excuses to those I have omitted.
\end{frame}

\begin{frame}[plain,fragile]
\frametitle{And sponsors}

\begin{enumerate}
\item National Science Foundation, US (various grants)

\item Department of Energy, US (various grants)

\item Research Council of Norway (various grants) and my employers University of Oslo and Michigan State University
\end{enumerate}

\noindent
\end{frame}

\begin{frame}[plain,fragile]
\frametitle{AI/ML and some statements you may have heard (and what do they mean?)}

\begin{enumerate}
\item Fei-Fei Li on ImageNet: \textbf{map out the entire world of objects} (\href{{https://cacm.acm.org/news/219702-the-data-that-transformed-ai-research-and-possibly-the-world/fulltext}}{The data that transformed AI research})

\item Russell and Norvig in their popular textbook: \textbf{relevant to any intellectual task; it is truly a universal field} (\href{{http://aima.cs.berkeley.edu/}}{Artificial Intelligence, A modern approach})

\item Woody Bledsoe puts it more bluntly: \textbf{in the long run, AI is the only science} (quoted in Pamilla McCorduck, \href{{https://www.pamelamccorduck.com/machines-who-think}}{Machines who think})
\end{enumerate}

\noindent
If you wish to have a critical read on AI/ML from a societal point of view, see \href{{https://www.katecrawford.net/}}{Kate Crawford's recent text Atlas of AI}. 

\textbf{Here: with AI/ML we intend a collection of machine learning methods with an emphasis on statistical learning and data analysis}
\end{frame}

\begin{frame}[plain,fragile]
\frametitle{Types of machine learning}

\begin{block}{}
The approaches to machine learning are many, but are often split into two main categories. 
In \emph{supervised learning} we know the answer to a problem,
and let the computer deduce the logic behind it. On the other hand, \emph{unsupervised learning}
is a method for finding patterns and relationship in data sets without any prior knowledge of the system.

An important  third category is  \emph{reinforcement learning}. This is a paradigm 
of learning inspired by behavioural psychology, where learning is achieved by trial-and-error, 
solely from rewards and punishment.
\end{block}
\end{frame}

\begin{frame}[plain,fragile]
\frametitle{Main categories}

\begin{block}{}
Another way to categorize machine learning tasks is to consider the desired output of a system.
Some of the most common tasks are:

\begin{itemize}
  \item Classification: Outputs are divided into two or more classes. The goal is to   produce a model that assigns inputs into one of these classes. An example is to identify  digits based on pictures of hand-written ones. Classification is typically supervised learning.

  \item Regression: Finding a functional relationship between an input data set and a reference data set.   The goal is to construct a function that maps input data to continuous output values.

  \item Clustering: Data are divided into groups with certain common traits, without knowing the different groups beforehand.  It is thus a form of unsupervised learning.
\end{itemize}

\noindent
\end{block}
\end{frame}

\begin{frame}[plain,fragile]
\frametitle{The plethora  of machine learning algorithms/methods}

\begin{enumerate}
\item Deep learning: Neural Networks (NN), Convolutional NN, Recurrent NN, Boltzmann machines, autoencoders and variational autoencoders  and generative adversarial networks, stable diffusion and many more generative models

\item Bayesian statistics and Bayesian Machine Learning, Bayesian experimental design, Bayesian Regression models, Bayesian neural networks, Gaussian processes and much more

\item Dimensionality reduction (Principal component analysis), Clustering Methods and more

\item Ensemble Methods, Random forests, bagging and voting methods, gradient boosting approaches 

\item Linear and logistic regression, Kernel methods, support vector machines and more

\item Reinforcement Learning; Transfer Learning and more 
\end{enumerate}

\noindent
\end{frame}

\begin{frame}[plain,fragile]
\frametitle{Example of generative modeling, \href{{https://www.oreilly.com/library/view/generative-deep-learning/9781098134174/ch01.html}}{taken from Generative Deep Learning by David Foster}}

\vspace{6mm}

% inline figure
\centerline{\includegraphics[width=1.0\linewidth]{figures/generativelearning.png}}

\vspace{6mm}
\end{frame}

\begin{frame}[plain,fragile]
\frametitle{Example of discriminative modeling, \href{{https://www.oreilly.com/library/view/generative-deep-learning/9781098134174/ch01.html}}{taken from Generative Deeep Learning by David Foster}}

\vspace{6mm}

% inline figure
\centerline{\includegraphics[width=1.0\linewidth]{figures/standarddeeplearning.png}}

\vspace{6mm}
\end{frame}

\begin{frame}[plain,fragile]
\frametitle{Taxonomy of generative deep learning, \href{{https://www.oreilly.com/library/view/generative-deep-learning/9781098134174/ch01.html}}{taken from Generative Deep Learning by David Foster}}

\vspace{6mm}

% inline figure
\centerline{\includegraphics[width=1.0\linewidth]{figures/generativemodels.png}}

\vspace{6mm}
\end{frame}

\begin{frame}[plain,fragile]
\frametitle{Good books with hands-on material and codes}

\begin{block}{}
\begin{itemize}
\item \href{{https://sebastianraschka.com/blog/2022/ml-pytorch-book.html}}{Sebastian Rashcka et al, Machine learning with Sickit-Learn and PyTorch}

\item \href{{https://www.oreilly.com/library/view/generative-deep-learning/9781098134174/ch01.html}}{David Foster, Generative Deep Learning with TensorFlow}

\item \href{{https://github.com/PacktPublishing/Hands-On-Generative-AI-with-Python-and-TensorFlow-2}}{Bali and Gavras, Generative AI with Python and TensorFlow 2}
\end{itemize}

\noindent
\end{block}

All three books have GitHub addresses from where  one can download all codes. We will borrow most of the material from these three texts as well as 
from Goodfellow, Bengio and Courville's text \href{{https://www.deeplearningbook.org/}}{Deep Learning}
\end{frame}

\begin{frame}[plain,fragile]
\frametitle{What are the basic Machine Learning ingredients?}

\begin{block}{}
Almost every problem in ML and data science starts with the same ingredients:
\begin{itemize}
\item The dataset $\bm{x}$ (could be some observable quantity of the system we are studying)

\item A model which is a function of a set of parameters $\bm{\alpha}$ that relates to the dataset, say a likelihood  function $p(\bm{x}\vert \bm{\alpha})$ or just a simple model $f(\bm{\alpha})$

\item A so-called \textbf{loss/cost/risk} function $\mathcal{C} (\bm{x}, f(\bm{\alpha}))$ which allows us to decide how well our model represents the dataset. 
\end{itemize}

\noindent
We seek to minimize the function $\mathcal{C} (\bm{x}, f(\bm{\alpha}))$ by finding the parameter values which minimize $\mathcal{C}$. This leads to  various minimization algorithms. It may surprise many, but at the heart of all machine learning algortihms there is an optimization problem. 
\end{block}
\end{frame}

\begin{frame}[plain,fragile]
\frametitle{Low-level machine learning, the family of ordinary least squares methods}

Our data which we want to apply a machine learning method on, consist
of a set of inputs $\bm{x}^T=[x_0,x_1,x_2,\dots,x_{n-1}]$ and the
outputs we want to model $\bm{y}^T=[y_0,y_1,y_2,\dots,y_{n-1}]$.
We assume  that the output data can be represented (for a regression case) by a continuous function $f$
through
\[
\bm{y}=f(\bm{x})+\bm{\epsilon}.
\]
\end{frame}

\begin{frame}[plain,fragile]
\frametitle{Setting up the equations}

In linear regression we approximate the unknown function with another
continuous function $\tilde{\bm{y}}(\bm{x})$ which depends linearly on
some unknown parameters
$\bm{\theta}^T=[\theta_0,\theta_1,\theta_2,\dots,\theta_{p-1}]$.

The input data can be organized in terms of a so-called design matrix 
with an approximating function $\bm{\tilde{y}}$ 
\[
\bm{\tilde{y}}= \bm{X}\bm{\theta},
\]
\end{frame}

\begin{frame}[plain,fragile]
\frametitle{The objective/cost/loss function}

The  simplest approach is the mean squared error
\[
C(\bm{\Theta})=\frac{1}{n}\sum_{i=0}^{n-1}\left(y_i-\tilde{y}_i\right)^2=\frac{1}{n}\left\{\left(\bm{y}-\bm{\tilde{y}}\right)^T\left(\bm{y}-\bm{\tilde{y}}\right)\right\},
\]
or using the matrix $\bm{X}$ and in a more compact matrix-vector notation as
\[
C(\bm{\Theta})=\frac{1}{n}\left\{\left(\bm{y}-\bm{X}\bm{\theta}\right)^T\left(\bm{y}-\bm{X}\bm{\theta}\right)\right\}.
\]
This function represents one of many possible ways to define the so-called cost function.
\end{frame}

\begin{frame}[plain,fragile]
\frametitle{Training solution}

Optimizing with respect to the unknown parameters $\theta_j$ we get 
\[
\bm{X}^T\bm{y} = \bm{X}^T\bm{X}\bm{\theta},  
\]
and if the matrix $\bm{X}^T\bm{X}$ is invertible we have the optimal values
\[
\hat{\bm{\theta}} =\left(\bm{X}^T\bm{X}\right)^{-1}\bm{X}^T\bm{y}.
\]

We say we 'learn' the unknown parameters $\bm{\theta}$ from the last equation.
\end{frame}

\begin{frame}[plain,fragile]
\frametitle{Ridge and LASSO Regression}

Our optimization problem is
\[
{\displaystyle \min_{\bm{\theta}\in {\mathbb{R}}^{p}}}\frac{1}{n}\left\{\left(\bm{y}-\bm{X}\bm{\theta}\right)^T\left(\bm{y}-\bm{X}\bm{\theta}\right)\right\}.
\]
or we can state it as
\[
{\displaystyle \min_{\bm{\theta}\in
{\mathbb{R}}^{p}}}\frac{1}{n}\sum_{i=0}^{n-1}\left(y_i-\tilde{y}_i\right)^2=\frac{1}{n}\vert\vert \bm{y}-\bm{X}\bm{\theta}\vert\vert_2^2,
\]
where we have used the definition of  a norm-2 vector, that is
\[
\vert\vert \bm{x}\vert\vert_2 = \sqrt{\sum_i x_i^2}. 
\]
\end{frame}

\begin{frame}[plain,fragile]
\frametitle{From OLS to Ridge and Lasso}

By minimizing the above equation with respect to the parameters
$\bm{\theta}$ we could then obtain an analytical expression for the
parameters $\bm{\theta}$.  We can add a regularization parameter $\lambda$ by
defining a new cost function to be optimized, that is

\[
{\displaystyle \min_{\bm{\theta}\in
{\mathbb{R}}^{p}}}\frac{1}{n}\vert\vert \bm{y}-\bm{X}\bm{\theta}\vert\vert_2^2+\lambda\vert\vert \bm{\theta}\vert\vert_2^2
\]

which leads to the Ridge regression minimization problem where we
require that $\vert\vert \bm{\theta}\vert\vert_2^2\le t$, where $t$ is
a finite number larger than zero. We do not include such a constraints in the discussions here.
\end{frame}

\begin{frame}[plain,fragile]
\frametitle{Lasso regression}

Defining

\[
C(\bm{X},\bm{\theta})=\frac{1}{n}\vert\vert \bm{y}-\bm{X}\bm{\theta}\vert\vert_2^2+\lambda\vert\vert \bm{\theta}\vert\vert_1,
\]
we have a new optimization equation
\[
{\displaystyle \min_{\bm{\theta}\in
{\mathbb{R}}^{p}}}\frac{1}{n}\vert\vert \bm{y}-\bm{X}\bm{\theta}\vert\vert_2^2+\lambda\vert\vert \bm{\theta}\vert\vert_1
\]
which leads to Lasso regression. Lasso stands for least absolute shrinkage and selection operator. 
Here we have defined the norm-1 as 
\[
\vert\vert \bm{x}\vert\vert_1 = \sum_i \vert x_i\vert. 
\]
\end{frame}

\begin{frame}[plain,fragile]
\frametitle{Lots of room for creativity}

\begin{block}{}
Not all the
algorithms and methods can be given a rigorous mathematical
justification, opening up thereby for experimenting
and trial and error and thereby exciting new developments. 
\end{block}

\begin{block}{}
A solid command of linear algebra, multivariate theory, 
probability theory, statistical data analysis, optimization algorithms, 
understanding errors and Monte Carlo methods is important in order to understand many of the 
various algorithms and methods. 
\end{block}

\textbf{Job market, a personal statement}: \href{{https://www.analyticsindiamag.com/top-countries-hiring-most-number-of-artificial-intelligence-machine-learning-experts/}}{A familiarity with ML is almost becoming a prerequisite for many of the most exciting employment opportunities}. And add quantum computing and there you are!
\end{frame}

\begin{frame}[plain,fragile]
\frametitle{Selected references}

\begin{block}{}
\begin{itemize}
\item \href{{https://arxiv.org/abs/1803.08823}}{Mehta et al.} and \href{{https://www.sciencedirect.com/science/article/pii/S0370157319300766?via%3Dihub}}{Physics Reports (2019)}.

\item \href{{https://link.aps.org/doi/10.1103/RevModPhys.91.045002}}{Machine Learning and the Physical Sciences by Carleo et al}

\item \href{{https://journals.aps.org/rmp/abstract/10.1103/RevModPhys.94.031003}}{Artificial Intelligence and Machine Learning in Nuclear Physics, Amber Boehnlein et al., Reviews Modern of Physics 94, 031003 (2022)} 

\item \href{{https://journals.aps.org/prresearch/pdf/10.1103/PhysRevResearch.5.033062}}{Dilute neutron star matter from neural-network quantum states by Fore et al, Physical Review Research 5, 033062 (2023)}

\item \href{{https://doi.org/10.48550/arXiv.2305.08831}}{Neural-network quantum states for ultra-cold Fermi gases, Jane Kim et al, Nature Physics Communcication, submitted}

\item \href{{https://doi.org/10.48550/arXiv.2305.07240}}{Message-Passing Neural Quantum States for the Homogeneous Electron Gas, Gabriel Pescia, Jane Kim et al.~arXiv.2305.07240,}

\item "Efficient solutions of fermionic systems using artificial neural networks, Nordhagen et al, Frontiers in Physics
\end{itemize}

\noindent
\end{block}
\end{frame}

\begin{frame}[plain,fragile]
\frametitle{Machine learning. A simple perspective on the interface between ML and Physics}

\vspace{6mm}

% inline figure
\centerline{\includegraphics[width=1.0\linewidth]{figures/mlimage.png}}

\vspace{6mm}
\end{frame}

\begin{frame}[plain,fragile]
\frametitle{ML in Nuclear  Physics (or any field in physics)}

\vspace{6mm}

% inline figure
\centerline{\includegraphics[width=1.0\linewidth]{figures/ML-NP.pdf}}

\vspace{6mm}
\end{frame}

\begin{frame}[plain,fragile]
\frametitle{Scientific Machine Learning}

An important and emerging field is what has been dubbed as scientific ML, see the article by Deiana et al "Applications and Techniques for Fast Machine Learning in Science, Big Data \textbf{5}, 787421 (2022):https://doi.org/10.3389/fdata.2022.787421"

\begin{block}{}
The authors discuss applications and techniques for fast machine
learning (ML) in science -- the concept of integrating power ML
methods into the real-time experimental data processing loop to
accelerate scientific discovery. The report covers three main areas

\begin{enumerate}
\item applications for fast ML across a number of scientific domains;

\item techniques for training and implementing performant and resource-efficient ML algorithms;

\item and computing architectures, platforms, and technologies for deploying these algorithms.
\end{enumerate}

\noindent
\end{block}
\end{frame}

\begin{frame}[plain,fragile]
\frametitle{ML for detectors}

\vspace{6mm}

% inline figure
\centerline{\includegraphics[width=1.0\linewidth]{figures/detectors.png}}

\vspace{6mm}
\end{frame}

\begin{frame}[plain,fragile]
\frametitle{Physics driven Machine Learning}

Another hot topic is what has loosely been dubbed \textbf{Physics-driven deep learning}. See the recent work on \href{{https://www.nature.com/articles/s42256-021-00302-5}}{Learning nonlinear operators via DeepONet based on the universal approximation theorem of operators, Nature Machine Learning, vol 3, 218 (2021)}.

\begin{block}{From their abstract }
A less known but powerful result is that an NN with a single hidden layer can accurately approximate any nonlinear continuous operator. This universal approximation theorem of operators is suggestive of the structure and potential of deep neural networks (DNNs) in learning continuous operators or complex systems from streams of scattered data. ...  We demonstrate that DeepONet can learn various explicit operators, such as integrals and fractional Laplacians, as well as implicit operators that represent deterministic and stochastic differential equations. 
\end{block}
\end{frame}

\begin{frame}[plain,fragile]
\frametitle{And more}

\begin{block}{}
\begin{itemize}
\item An important application of AI/ML methods is to improve the estimation of bias or uncertainty due to the introduction of or lack of physical constraints in various theoretical models.

\item In theory, we expect to use AI/ML algorithms and methods to improve our knowledge about  correlations of physical model parameters in data for quantum many-body systems. Deep learning methods show great promise in circumventing the exploding dimensionalities encountered in quantum mechanical many-body studies. 

\item Merging a frequentist approach (the standard path in ML theory) with a Bayesian approach, has the potential to infer better probabilitity distributions and error estimates. 

\item Machine Learning and Quantum Computing is a very interesting avenue to explore.
\end{itemize}

\noindent
\end{block}
\end{frame}

\begin{frame}[plain,fragile]
\frametitle{Argon-46 by Solli et al., NIMA 1010, 165461 (2021)}

\begin{block}{}
Representations of two events from an 
Argon-46 experiment at MSU. Each row is one event in two projections,
where the color intensity of each point indicates higher charge values
recorded by the detector. The bottom row illustrates a carbon event with
a large fraction of noise, while the top row shows a proton event
almost free of noise. 
\end{block}

\vspace{6mm}

% inline figure
\centerline{\includegraphics[width=0.6\linewidth]{figures/examples_raw.png}}

\vspace{6mm}
\end{frame}

\begin{frame}[plain,fragile]
\frametitle{Why Feed Forward Neural Networks (FFNN)?}

According to the \emph{Universal approximation theorem}, a feed-forward
neural network with just a single hidden layer containing a finite
number of neurons can approximate a continuous multidimensional
function to arbitrary accuracy, assuming the activation function for
the hidden layer is a \textbf{non-constant, bounded and
monotonically-increasing continuous function}.
\end{frame}

\begin{frame}[plain,fragile]
\frametitle{Universal approximation theorem}

The universal approximation theorem plays a central role in deep
learning.  \href{{https://link.springer.com/article/10.1007/BF02551274}}{Cybenko (1989)} showed
the following:

\begin{block}{}
Let $\sigma$ be any continuous sigmoidal function such that
\[
\sigma(z) = \left\{\begin{array}{cc} 1 & z\rightarrow \infty\\ 0 & z \rightarrow -\infty \end{array}\right.
\]
Given a continuous and deterministic function $F(\bm{x})$ on the unit
cube in $d$-dimensions $F\in [0,1]^d$, $x\in [0,1]^d$ and a parameter
$\epsilon >0$, there is a one-layer (hidden) neural network
$f(\bm{x};\bm{\Theta})$ with $\bm{\Theta}=(\bm{W},\bm{b})$ and $\bm{W}\in
\mathbb{R}^{m\times n}$ and $\bm{b}\in \mathbb{R}^{n}$, for which
\[
\vert F(\bm{x})-f(\bm{x};\bm{\Theta})\vert < \epsilon \hspace{0.1cm} \forall \bm{x}\in[0,1]^d.
\]

\end{block}
\end{frame}

\begin{frame}[plain,fragile]
\frametitle{The approximation theorem in words}

\textbf{Any continuous function $y=F(\bm{x})$ supported on the unit cube in
$d$-dimensions can be approximated by a one-layer sigmoidal network to
arbitrary accuracy.}

\href{{https://www.sciencedirect.com/science/article/abs/pii/089360809190009T}}{Hornik (1991)} extended the theorem by letting any non-constant, bounded activation function to be included using that the expectation value
\[
\mathbb{E}[\vert F(\bm{x})\vert^2] =\int_{\bm{x}\in D} \vert F(\bm{x})\vert^2p(\bm{x})d\bm{x} < \infty.
\]
Then we have
\[
\mathbb{E}[\vert F(\bm{x})-f(\bm{x};\bm{\Theta})\vert^2] =\int_{\bm{x}\in D} \vert F(\bm{x})-f(\bm{x};\bm{\Theta})\vert^2p(\bm{x})d\bm{x} < \epsilon.
\]
\end{frame}

\begin{frame}[plain,fragile]
\frametitle{More on the general approximation theorem}

None of the proofs give any insight into the relation between the
number of of hidden layers and nodes and the approximation error
$\epsilon$, nor the magnitudes of $\bm{W}$ and $\bm{b}$.

Neural networks (NNs) have what we may call a kind of universality no matter what function we want to compute.

\begin{block}{}
It does not mean that an NN can be used to exactly compute any function. Rather, we get an approximation that is as good as we want. 
\end{block}
\end{frame}

\begin{frame}[plain,fragile]
\frametitle{Class of functions we can approximate}

\begin{block}{}
The class of functions that can be approximated are the continuous ones.
If the function $F(\bm{x})$ is discontinuous, it won't in general be possible to approximate it. However, an NN may still give an approximation even if we fail in some points.
\end{block}
\end{frame}

\begin{frame}[plain,fragile]
\frametitle{Illustration of a single perceptron model and an FFNN}

\begin{figure}[!ht]  % 
  \centerline{\includegraphics[width=0.7\linewidth]{figures/nns.png}}
  \caption{
  In a) we show a single perceptron model while in b) we dispay a network with two  hidden layers, an input layer and an output layer.
  }
\end{figure}
%\clearpage % flush figures
\end{frame}

\begin{frame}[plain,fragile]
\frametitle{Our network example, simple percepetron with one input}

As as simple example we define now a simple perceptron model with
all quantities given by scalars. We consider only one input variable
$x$ and one target value $y$.  We define an activation function
$\sigma_1$ which takes as input

\[
z_1 = w_1x+b_1,
\]
where $w_1$ is the weight and $b_1$ is the bias. These are the
parameters we want to optimize.  This output is then fed into the
\textbf{cost/loss} function, which we here for the sake of simplicity just
define as the squared error

\[
C(x;w_1,b_1)=\frac{1}{2}(a_1-y)^2.
\]
\end{frame}

\begin{frame}[plain,fragile]
\frametitle{Optimizing the parameters}

In setting up the feed forward and back propagation parts of the
algorithm, we need now the derivative of the various variables we want
to train.

We need
\[
\frac{\partial C}{\partial w_1} \hspace{0.1cm}\mathrm{and}\hspace{0.1cm}\frac{\partial C}{\partial b_1}. 
\]

Using the chain rule we find 
\[
\frac{\partial C}{\partial w_1}=\frac{\partial C}{\partial a_1}\frac{\partial a_1}{\partial z_1}\frac{\partial z_1}{\partial w_1}=(a_1-y)\sigma_1'x,
\]
and
\[
\frac{\partial C}{\partial b_1}=\frac{\partial C}{\partial a_1}\frac{\partial a_1}{\partial z_1}\frac{\partial z_1}{\partial b_1}=(a_1-y)\sigma_1',
\]
which we later will just define as
\[
\frac{\partial C}{\partial a_1}\frac{\partial a_1}{\partial z_1}=\delta_1.
\]
\end{frame}

\begin{frame}[plain,fragile]
\frametitle{Implementing the simple perceptron model}

In the example code here we implement the above equations (with explict
expressions for the derivatives) with just one input variable $x$ and
one output variable.  The target value $y=2x+1$ is a simple linear
function in $x$. Since this is a regression problem, we define the cost function to be proportional to the least squares error
\[
C(y,w_1,b_1)=\frac{1}{2}(a_1-y)^2,
\]
with $a_1$ the output from the network.





















































\begin{minted}[fontsize=\fontsize{9pt}{9pt},linenos=false,mathescape,baselinestretch=1.0,fontfamily=tt,xleftmargin=2mm]{python}
# import necessary packages
import numpy as np
import matplotlib.pyplot as plt

def feed_forward(x):
    # weighted sum of inputs to the output layer
    z_1 = x*output_weights + output_bias
    # Output from output node (one node only)
    # Here the output is equal to the input
    a_1 = z_1
    return a_1

def backpropagation(x, y):
    a_1 = feed_forward(x)
    # derivative of cost function
    derivative_cost = a_1 - y
    # the variable delta in the equations, note that output a_1 = z_1, its derivatives wrt z_o is thus 1
    delta_1 = derivative_cost
    # gradients for the output layer
    output_weights_gradient = delta_1*x
    output_bias_gradient = delta_1
    # The cost function is 0.5*(a_1-y)^2. This gives a measure of the error for each iteration
    return output_weights_gradient, output_bias_gradient

# ensure the same random numbers appear every time
np.random.seed(0)
# Input variable
x = 4.0
# Target values
y = 2*x+1.0

# Defining the neural network
n_inputs = 1
n_outputs = 1
# Initialize the network
# weights and bias in the output layer
output_weights = np.random.randn()
output_bias = np.random.randn()

# implementing a simple gradient descent approach with fixed learning rate
eta = 0.01
for i in range(40):
    # calculate gradients from back propagation
    derivative_w1, derivative_b1 = backpropagation(x, y)
    # update weights and biases
    output_weights -= eta * derivative_w1
    output_bias -= eta * derivative_b1
# our final prediction after training
ytilde = output_weights*x+output_bias
print(0.5*((ytilde-y)**2))


\end{minted}

Running this code gives us an acceptable results after some 40-50 iterations. Note that the results depend on the value of the learning rate.
\end{frame}

\begin{frame}[plain,fragile]
\frametitle{Central magic}

\href{{https://en.wikipedia.org/wiki/Automatic_differentiation}}{Automatic differentiation}
\end{frame}

\end{document}
